\documentclass[11pt]{article}
\usepackage[a4paper,margin=2cm]{geometry}
\usepackage[english]{babel}
\usepackage[utf8]{inputenc}
\usepackage[T1]{fontenc}
\linespread{1.3}
\parskip=12pt
\parindent=0pt
\usepackage{enumitem}
\usepackage{amsmath}
\usepackage{amsfonts}
\usepackage{graphicx}
\usepackage{amssymb}
\usepackage{hyperref}
\usepackage{amsthm}
\usepackage{color}


% Defining the question styles
\theoremstyle{definition}
\newtheorem{prob}{Problem}

% Custom commands
\newcommand{\E}{\mathbb{E}}
\newcommand{\Var}{\mathrm{Var}}
\newcommand{\Prob}{\mathbb{P}}

% declare a new theorem style
\newtheoremstyle{solution}%
{1pt}% Space above
{1pt}% Space below 
{\itshape\color{red}}% Body font
{}% Indent amount
{\bfseries\color{red}}% Theorem head font
{.}% Punctuation after theorem head
{.5em}% Space after theorem head
{}% Theorem head spec (can be left empty, meaning ‘normal’)

\theoremstyle{solution}
\newtheorem*{solution}{Solution}

% --- Code starts here ---
\begin{document}
	\begin{center}
		{\Large{\textbf{Lista IV - Métodos Numéricos}}}\\
		\vspace{0.2cm}
		EPGE - 2018\\
		Professor: Cezar Santos\\
		Aluno: Raul Guarini Riva
	\end{center}
	

O código principal da lista está no arquivo \texttt{ps4.m}. Como nas outras listas, utilizei minha função \texttt{tauchen\_ar1} para realizar a discretização desejada do grid em 9 pontos. O problema do fazendeiro consiste em escolher consumo e número de cabras que serão guardadas em cada instante do tempo. Portanto, as variáveis de estado são o choque de dotação atual $z$ e o nível de cabras estocadas no presente $a$. As variáveis de controle são o consumo $c$ e o número de cabras a serem estocadas $a'$. A equação funcional que o fazendeiro tenta resolver, dado $q$, é dada por:
\begin{gather*}
	V(z, a) = \max\limits_{c, a'} \{ u(c) + \beta\E(V(z', a')|z)\}\\
	\text{s.t.} \quad c + qa' = e^z + a
\end{gather*}
Para a discretização do espaço de ativos, precisamos de um limite de endividamento adequado. Em geral, poderíamos utilizar o limite natural de endividamento. Contudo, neste contexto rural, parece fazer sentido definir como limite inferior para o número de cabras estocadas o valor zero.

\end{document}
