\documentclass[11pt]{article}
\usepackage[a4paper,margin=2cm]{geometry}
\usepackage[brazilian]{babel}
\usepackage[utf8]{inputenc}
\usepackage[T1]{fontenc}
\linespread{1.3}
\parskip=12pt
\parindent=0pt
\usepackage{enumitem}
\usepackage{amsmath}
\usepackage{amsfonts}
\usepackage{graphicx}
\usepackage{amssymb}
\usepackage{hyperref}
\usepackage{amsthm}
\usepackage{color}


% Defining the question styles
\theoremstyle{definition}
\newtheorem{prob}{Problema}

% Custom commands
\newcommand{\E}{\mathbb{E}}
\newcommand{\Var}{\mathrm{Var}}
\newcommand{\Prob}{\mathbb{P}}

% declare a new theorem style
\newtheoremstyle{solution}%
{1pt}% Space above
{1pt}% Space below 
{\itshape\color{red}}% Body font
{}% Indent amount
{\bfseries\color{red}}% Theorem head font
{.}% Punctuation after theorem head
{.5em}% Space after theorem head
{}% Theorem head spec (can be left empty, meaning ‘normal’)

\theoremstyle{solution}
\newtheorem*{solution}{Solution}

% --- Code starts here ---
\begin{document}
	\begin{center}
		{\Large{\textbf{Lista III - Métodos Numéricos}}}\\
		\vspace{0.2cm}
		EPGE - 2018\\
		Professor: Cézar Santos\\
		Aluno: Raul Guarini Riva
	\end{center}
	
\begin{prob}
	Segui a mesma calibração da lista anterior e encontrei as funções política do consumo e do capital através da técnica de colocação aliada aos polinômios de Chebyshev. O código principal da lista está no arquivo \texttt{ps3.m}. O tempo de execução desta técnica de projeção espectral foi o menor dentre todas as técnicas até agora (desde a Lista 2): ao redor de 0.4 segundos.
	
	A discretização do processo estocástico da TFP foi feito através da técnica de Tauchen, da mesma maneira das outras listas. Utilizei 7 polinômios de Chebyshev para computar a projeção ($d = 6$, na notação dos slides). O problema é muito bem comportado e, portanto, não foi necessário $d$ muito grande. Testei dimensões maiores mas os valores correspondente de $\gamma_{j}$ rapidamente iam para zero, indicado não ser necessário mais graus de liberdade para a projeção.
	
	A função \texttt{chebyshev\_poly} computa raízes de um polinômio de Chebyshev de ordem $k$ e pode avaliá-lo em qualquer ponto de $[-1,1]$. A função \texttt{C\_proj} computa a projeção da função política do consumo dado um vetor $(d+1)$-dimensional $\gamma$ e um grid no qual o valor do capital de interesse se encontra. Finalmente a função \texttt{risk\_function} computa o resíduo\footnote{Não sei bem o motivo, mas de alguma forma achei que o $R(\gamma, K)$ dos slides eram ``Risk'' e não ``Residual''. Implementei tudo e usei a função várias vezes, só notei no final a confusão. Achei que iria semear bugs se trocasse o nome, então deixei desse jeito. Desculpa!} da projeção. É esta a função que define o sistema de $(d+1)$ equações que a função do MATLAB \texttt{fsolve} resolve. A próxima figura ilustra as funções política e Euler Errors.
	\begin{figure}[ht!]
		\centering
		\includegraphics[scale=0.25]{item1.png}
	\end{figure}
	
	Assim como antes, a função política do capital é praticamente linear e não varia muito de acordo com a TFP. Entretanto, níveis maiores de TFP levam a níveis maiores de capital no próximo período, dado o mesmo capital inicial. Este tipo de monotonicidade também é verificada na função política do consumo. Nota-se que esta tem maior curvatura do que a função política do capital.
	
	Curiosamente, o aspecto das curvas de Euler Errors é diferente daquele encontrado, por exemplo, na implementação do grid endógeno na Lista 2. As curvas do erro são basicamente funções constantes de $k$ mas apresentam um comportamento interessante em $z$. Prestando atenção na legenda, vemos que o menor erro de aproximação foi alcançado no valor central do grid de $iz = 4$ e as outras curvas parecem estar dispostas duas a duas de acordo com a distância ao centro. Quanto mais próximo a este, menor o erro de aproximação. Apesar desta observação, não consegui interpretar este resultado.
\end{prob}

\end{document}
